\documentclass{article}

\usepackage[ngerman]{babel}
\usepackage[utf8]{inputenc}
\usepackage[T1]{fontenc}
\usepackage{hyperref}
\usepackage{csquotes}

\usepackage[
    backend=biber,
    style=apa,
    sortlocale=de_DE,
    natbib=true,
    url=false,
    doi=false,
    sortcites=true,
    sorting=nyt,
    isbn=false,
    hyperref=true,
    backref=false,
    giveninits=false,
    eprint=false]{biblatex}
\addbibresource{../references/bibliography.bib}

\title{Notizen zum Projekt Data Ethics}
\author{Luisa Rudin}
\date{\today}

\begin{document}
\maketitle

\abstract{
    Wie kann die Diskriminierung durch KI-Algorithmen verhindert werden und welche ethischen Maßnahmen sind notwendig?
}

\tableofcontents

\section{Wie ensteht die Diskriminierung?}
Eine KI kann diskriminierend gegenüber bestimmten Personengruppen werden, wenn die Entwickler bestimmte stereotype Positionen in den Algorithmus einbauen, die zu Diskriminierung führen. Dies kann auch passieren, wenn bestimmte Variablen ausgewählt werden, die bereits diskriminierende Vorurteile enthalten. Diskriminierung kann auch während des Trainings und der Anwendung der KI entstehen, wenn nicht darauf geachtet wird, dass menschliche Voreingenommenheiten ausgeschlossen werden. Es ist wichtig, dass bei der Entwicklung von KI-Systemen darauf geachtet wird, Diskriminierung zu verhindern und alle geschützten Merkmale angemessen zu berücksichtigen.


\section{Was tut die Behörde?}
Die Antidiskriminierungsstelle des Bundes kann gegen die Diskriminierung von KI vorgehen, indem sie Menschen unterstützt, die von Diskriminierung betroffen sind, informiert, wissenschaftliche Untersuchungen durchführt und an den Deutschen Bundestag berichtet.
\input{section_ai.tex}

\printbibliography

\end{document}
