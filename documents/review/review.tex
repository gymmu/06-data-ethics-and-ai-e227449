\documentclass{article}

\usepackage[ngerman]{babel}
\usepackage[utf8]{inputenc}
\usepackage[T1]{fontenc}
\usepackage{hyperref}
\usepackage{csquotes}

\usepackage[
    backend=biber,
    style=apa,
    sortlocale=de_DE,
    natbib=true,
    url=false,
    doi=false,
    sortcites=true,
    sorting=nyt,
    isbn=false,
    hyperref=true,
    backref=false,
    giveninits=false,
    eprint=false]{biblatex}
\addbibresource{../references/bibliography.bib}

\title{Review des Papers "Ist es ethisch korrekt, wenn man in der Öffentlichkeit Siri eingeschaltet hat?" von Ann-Cathrine Böller}
\author{Luisa Rudin}
\date{\today}

\begin{document}
\maketitle

\abstract{
    Dies ist ein Review der Arbeit zum Thema Ist es ethisch korrekt, wenn man in der Öffentlichkeit Siri eingeschaltet hat? von Ann-Cathrine Böller.
}

\section{Review}
Der Aufsatz ist professionell und informativ geschrieben. Er ist durch die zwei Bilder, welche beide eine gute Grösse haben, visuell ansprechend. Sie sind ebenfalls passend zu Thema und gut angeordnet. Die verschiedenen Abschnitte machen das Dokument sehr übersichtlich. Es gibt eine Einleitung zu dem Thema, in welcher nicht nur die KI beschrieben wird, was einen Überblickt über den Inhalt des Aufsatzes gibt. Die KI wurde ausführlich beschrieben und man kann dem Prinzip des Trainings folgen. Inhaltlich stimmt fast Alles; Dass Siri zwingend mit einer weiblichen Stimme antwortet, stimmt nicht immer. Man kann die Stimme durchaus in den Einstellungen ändern. Es stimmt allerdings, dass man die Siristimme eher unter der Weiblichen kennt. Ausserdem kann die Siri einfache, allgemeine Fragen durchaus beantworten kann, was auch die Seite https://support.apple.com/de-de/guide/homepod/apd1ac8031f8/homepod bestätigt. Zudem kann der Vorschlag im Abschnitt 7, dass man die Siri-Daten regelmässig löschen sollte, nicht ausgeführt werden, da es keine solche Funktion gibt. Man könnte jedoch den Verlauf der Diktierfunktion löschen. Wenn das noch spezifiziert wird, wäre die Information korrekter. Dies bestätigt die Quelle https://support.apple.com/de-ch/guide/mac-help/mchlf55961c0/mac. Es gibt ein oder zwei grammatikalische Fehler, welche allerdings kaum auffallen und nicht gravierend sind. 

\section{Verbesserungsvorschläge}
Man hätte vielleicht aus den beiden Abschnitten, welche die KI beschreiben, ein Kapitel machen können. So wäre es noch etwas übersichtlicher. Ausserde wäre es gut, wenn es ein paar Zitate im Text hätte. So würde der Text glaubwürdiger und hochwertiger herüber kommen. Der Titel ist sehr lang. Wenn man diesen etwas kürzer gestalten würde, steigt die Lust zum lesen. Die Schriftart ist überall gleich und somit monoton. Diese könnte ganz einfach geändert werden und würde das Dokument etwas lebendiger wirken lassen. 

\section{Fazit}
Der Text ist ansprechend und informativ. Man kann dem Text gut folgen und hat nach dem Durchlesen etwas dazu gelernt. Durch die Bilder wirkt der Aufsatz freundlich und vielverprechend. Insgesamt hast du das sehr gut gemacht, Ann-Cathrine! 


\end{document}
